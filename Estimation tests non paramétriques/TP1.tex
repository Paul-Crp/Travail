% Options for packages loaded elsewhere
\PassOptionsToPackage{unicode}{hyperref}
\PassOptionsToPackage{hyphens}{url}
%
\documentclass[
]{article}
\usepackage{amsmath,amssymb}
\usepackage{lmodern}
\usepackage{iftex}
\ifPDFTeX
  \usepackage[T1]{fontenc}
  \usepackage[utf8]{inputenc}
  \usepackage{textcomp} % provide euro and other symbols
\else % if luatex or xetex
  \usepackage{unicode-math}
  \defaultfontfeatures{Scale=MatchLowercase}
  \defaultfontfeatures[\rmfamily]{Ligatures=TeX,Scale=1}
\fi
% Use upquote if available, for straight quotes in verbatim environments
\IfFileExists{upquote.sty}{\usepackage{upquote}}{}
\IfFileExists{microtype.sty}{% use microtype if available
  \usepackage[]{microtype}
  \UseMicrotypeSet[protrusion]{basicmath} % disable protrusion for tt fonts
}{}
\makeatletter
\@ifundefined{KOMAClassName}{% if non-KOMA class
  \IfFileExists{parskip.sty}{%
    \usepackage{parskip}
  }{% else
    \setlength{\parindent}{0pt}
    \setlength{\parskip}{6pt plus 2pt minus 1pt}}
}{% if KOMA class
  \KOMAoptions{parskip=half}}
\makeatother
\usepackage{xcolor}
\usepackage[margin=1in]{geometry}
\usepackage{graphicx}
\makeatletter
\def\maxwidth{\ifdim\Gin@nat@width>\linewidth\linewidth\else\Gin@nat@width\fi}
\def\maxheight{\ifdim\Gin@nat@height>\textheight\textheight\else\Gin@nat@height\fi}
\makeatother
% Scale images if necessary, so that they will not overflow the page
% margins by default, and it is still possible to overwrite the defaults
% using explicit options in \includegraphics[width, height, ...]{}
\setkeys{Gin}{width=\maxwidth,height=\maxheight,keepaspectratio}
% Set default figure placement to htbp
\makeatletter
\def\fps@figure{htbp}
\makeatother
\setlength{\emergencystretch}{3em} % prevent overfull lines
\providecommand{\tightlist}{%
  \setlength{\itemsep}{0pt}\setlength{\parskip}{0pt}}
\setcounter{secnumdepth}{-\maxdimen} % remove section numbering
\ifLuaTeX
  \usepackage{selnolig}  % disable illegal ligatures
\fi
\IfFileExists{bookmark.sty}{\usepackage{bookmark}}{\usepackage{hyperref}}
\IfFileExists{xurl.sty}{\usepackage{xurl}}{} % add URL line breaks if available
\urlstyle{same} % disable monospaced font for URLs
\hypersetup{
  pdftitle={TP1 : Premiers pas pour la modélisation des séries temporelles},
  hidelinks,
  pdfcreator={LaTeX via pandoc}}

\title{TP1 : Premiers pas pour la modélisation des séries temporelles}
\author{}
\date{\vspace{-2.5em}}

\begin{document}
\maketitle

\#EXO 1

\hypertarget{section}{%
\subsection{1)}\label{section}}

View(sleep)

p \textless- ggplot(sleep, aes(x=group, y=extra)) + geom\_boxplot()

print(p)

p + coord\_flip()

p

\hypertarget{section-1}{%
\subsection{2)}\label{section-1}}

attach(sleep)

t.test(sleep,alternative = ``less'')

sleep\(extra sleep\)group

shapiro.test(extra{[}group==1{]}) shapiro.test(extra{[}group==2{]})

detach(sleep)

shapiro.test(sleep\(extra[sleep\)group==1{]})
shapiro.test(sleep\(extra[sleep\)group==2{]})

\hypertarget{section-2}{%
\subsection{3)}\label{section-2}}

attach(sleep)

bartlett.test(extra\textasciitilde group)

detach(extra)

bartlett.test(sleep\(extra~sleep\)group)

\hypertarget{section-3}{%
\subsection{4)}\label{section-3}}

attach(sleep)

t.test(extra{[}group==1{]},extra{[}group==2{]},alternative =
``less'',var.equal=TRUE)

detach(sleep)

\hypertarget{ne-pas-oublier-le-var.equalt-pour-faire-un-test-de-student-et-pas-la-variante-de-welch-degruxe9-de-libertuxe9-ruxe9el-contre-entiuxe8re}{%
\subsubsection{Ne pas oublier le var.equal=T pour faire un test de
Student et pas la variante de Welch (degré de liberté réel contre
entière)}\label{ne-pas-oublier-le-var.equalt-pour-faire-un-test-de-student-et-pas-la-variante-de-welch-degruxe9-de-libertuxe9-ruxe9el-contre-entiuxe8re}}

\hypertarget{il-ny-a-pas-de-choix-aluxe9atoire-de-population}{%
\subsection{5) Il n'y a pas de choix aléatoire de
population}\label{il-ny-a-pas-de-choix-aluxe9atoire-de-population}}

\#EXO 2

r=data.frame(Gr1=c(35, 40, 12, 15, 21, 14, 46, 10, 28, 48, 16, 30, 32,
48, 31, 22, 12, 39, 19, 25),Gr2=c(2, 27, 38, 31, 1, 19, 1, 34, 3, 1, 2,
3, 2, 1, 2, 1, 3, 29, 37, 2))

View(r)

r

rr \textless- ggplot(r, aes(x=Gr1, y=Gr2)) + geom\_boxplot()

print(rr)

r + coord\_flip()

r

attach(r)

t.test(r,alternative = ``less'')

sleep\(extra sleep\)group

shapiro.test(Gr1)

shapiro.test(Gr2)

bartlett.test(Gr1\textasciitilde Gr2)

detach(r)

v=data.frame(temps=c(35, 40, 12, 15, 21, 14, 46, 10, 28, 48, 16, 30, 32,
48, 31, 22, 12, 39, 19, 25, 2, 27, 38, 31, 1, 19, 1, 34, 3, 1, 2, 3, 2,
1, 2, 1, 3, 29, 37, 2),
group=c(1,1,1,1,1,1,1,1,1,1,1,1,1,1,1,1,1,1,1,1,2,2,2,2,2,2,2,2,2,2,2,2,2,2,2,2,2,2,2,2))

attach(v)

shapiro.test(temps{[}group==1{]}) shapiro.test(temps{[}group==2{]})

\#\#\#Pas de normalité sur le second groupe

bartlett.test(temps\textasciitilde group)

detach(v)

vv \textless- ggplot(v, aes(x=group, y=temps, groubp=20)) +
geom\_boxplot()

print(vv)

vv + coord\_flip()

v

boxplot(temps{[}group==1{]},temps{[}group==2{]})

\end{document}
